
\index{exit codes!of condor\_shadow}
When a \Condor{shadow} daemon exits, 
the \Condor{shadow} exit code is recorded in the \Condor{schedd} log,
and it identifies why the job exited. 
Prose in the log appears of the form
\begin{verbatim}
Shadow pid XXXXX for job XX.X exited with status YYY 
\end{verbatim}
where \Expr{YYY} is the exit code,  or
\begin{verbatim}
Shadow pid XXXXX for job XX.X reports job exit reason 100. 
\end{verbatim}
where the exit code is the value 100.
Table~\ref{table:shadow-exit-codes} lists these codes.
% shadow error codes
\begin{center}
\begin{table}[ht]
\caption{\label{table:shadow-exit-codes}\Condor{shadow} Exit Codes}
\begin{tabular}{|c|l|l|} \hline
\emph{Value} & \emph{Error Name} & \emph{Description} \\ \hline \hline
4   &   JOB\_EXCEPTION    & the job exited with an exception \\ \hline
44  &   DPRINTF\_ERROR    & there was a fatal error with dprintf() \\ \hline
100 &   JOB\_EXITED       & the job exited (not killed)  \\ \hline
101 &   JOB\_CKPTED       & the job did produce a checkpoint  \\ \hline
102 &   JOB\_KILLED       & the job was killed     \\ \hline
103 &   JOB\_COREDUMPED   & the job was killed and a core file was produced  \\ \hline
105 &   JOB\_NO\_MEM      & not enough memory to start the condor\_shadow \\ \hline
106 &   JOB\_SHADOW\_USAGE & incorrect arguments to condor\_shadow \\ \hline
107 &   JOB\_NOT\_CKPTED  & the job vacated without a checkpoint \\ \hline
107 &   JOB\_SHOULD\_REQUEUE  & same number as JOB\_NOT\_CKPTED, \\
    & &                       to achieve the same behavior.  \\
    & &                       This exit code implies that we want \\
    & &                       the job to be put back in the job queue \\
    & &                       and run again. \\ \hline
108 &   JOB\_NOT\_STARTED  & can not connect to the \Condor{startd} or request refused \\ \hline
109 &   JOB\_BAD\_STATUS  & job status != RUNNING on start up \\ \hline
110 &   JOB\_EXEC\_FAILED & exec failed for some reason other than ENOMEM \\ \hline
111 &   JOB\_NO\_CKPT\_FILE & there is no checkpoint file (as it was lost) \\ \hline
112 &   JOB\_SHOULD\_HOLD & the job should be put on hold \\ \hline
113 &   JOB\_SHOULD\_REMOVE & the job should be removed \\ \hline
114 &   JOB\_MISSED\_DEFERRAL\_TIME & the job goes on hold, because it did not run within the \\
    &                         & specified window of time \\  \hline
115 &   JOB\_EXITED\_AND\_CLAIM\_CLOSING & the job exited (not killed) but the \Condor{startd} \\
    &                         & is not accepting any more jobs on this claim \\  \hline
\end{tabular}
\end{table}
\end{center}

%\afterpage{\clearpage}

\index{log files!event codes for jobs}
Table~\ref{table:user-log-event-codes} lists codes that appear as 
the first field within a job event log file.
See more detailed descriptions of these values in
section~\ref{sec:job-log-events}.
\begin{center}
\begin{table}[ht]
\caption{\label{table:user-log-event-codes}Event Codes in a Job Event Log}
\begin{tabular}{|l|c|} \hline
\emph{Event Code} & \emph{Description}   \\ \hline \hline
000   &   Submit  \\ \hline
001   &   Execute  \\ \hline
002   &   Executable error  \\ \hline
003   &   Checkpointed  \\ \hline
004   &   Job evicted  \\ \hline
005   &   Job terminated  \\ \hline
006   &   Image size  \\ \hline
007   &   Shadow exception  \\ \hline
008   &   Generic  \\ \hline
009   &   Job aborted  \\ \hline
010  &   Job suspended  \\ \hline
011  &   Job unsuspended  \\ \hline
012  &   Job held  \\ \hline
013  &   Job released  \\ \hline
014  &   Node execute  \\ \hline
015  &   Node terminated  \\ \hline
016  &   Post script terminated  \\ \hline
017  &   Globus submit (no longer used)  \\ \hline
018  &   Globus submit failed  \\ \hline
019  &   Globus resource up (no longer used)  \\ \hline
020  &   Globus resource down (no longer used)  \\ \hline
021  &   Remote error  \\ \hline
022  &   Job disconnected  \\ \hline
023  &   Job reconnected  \\ \hline
024  &   Job reconnect failed  \\ \hline
025  &   Grid resource up \\ \hline
026  &   Grid resource down \\ \hline
027  &   Grid submit \\ \hline
028  &   Job ClassAd attribute values added to event log  \\ \hline
029  &   Job status unknown \\ \hline
030  &   Job status known \\ \hline
031  &   Grid job stage in \\ \hline
032  &   Grid job stage out \\ \hline
033  &   Job ClassAd attribute update \\ \hline
034  &   DAGMan PRE\_SKIP defined \\ \hline
\end{tabular}
\end{table}
\end{center}


\begin{center}
\begin{table}[ht]
\caption{\label{well-known-port-numbers}Well-Known Port Numbers}
\begin{tabular}{|l|c|} \hline
\emph{Server} & \emph{Port Number}   \\ \hline \hline
\Condor{negotiator}   &   9614 (obsolete, now dynamically allocated)   \\ \hline
\Condor{collector}    &   9618  \\ \hline
GT2 gatekeeper        &   2119  \\ \hline
gridftp               &   2811  \\ \hline
GT4 web services      &   8443  \\ \hline
\end{tabular}
\end{table}
\end{center}


\begin{center}
\begin{table}[ht]
\caption{\label{daemoncore-commands}DaemonCore Commands}
\begin{tabular}{|l|c|} \hline
\emph{Number} & \emph{Name}   \\ \hline \hline
60000  &   DC\_RAISESIGNAL                 \\ \hline
60001  &   DC\_PROCESSEXIT                 \\ \hline
60002  &   DC\_CONFIG\_PERSIST             \\ \hline
60003  &   DC\_CONFIG\_RUNTIME             \\ \hline
60004  &   DC\_RECONFIG                    \\ \hline
60005  &   DC\_OFF\_GRACEFUL               \\ \hline
60006  &   DC\_OFF\_FAST                   \\ \hline
60007  &   DC\_CONFIG\_VAL                 \\ \hline
60008  &   DC\_CHILDALIVE                  \\ \hline
60009  &   DC\_SERVICEWAITPIDS             \\ \hline
60010  &   DC\_AUTHENTICATE                \\ \hline
60011  &   DC\_NOP                         \\ \hline
60012  &   DC\_RECONFIG\_FULL              \\ \hline
60013  &   DC\_FETCH\_LOG                  \\ \hline
60014  &   DC\_INVALIDATE\_KEY             \\ \hline
60015  &   DC\_OFF\_PEACEFUL               \\ \hline
60016  &   DC\_SET\_PEACEFUL\_SHUTDOWN     \\ \hline
60017  &   DC\_TIME\_OFFSET                \\ \hline
60018  &   DC\_PURGE\_LOG                  \\ \hline
\end{tabular}
\end{table}
\end{center}


\begin{center}
\begin{table}[ht]
\caption{\label{daemon-exit-codes}DaemonCore Daemon Exit Codes}
\begin{tabular}{|l|c|} \hline
\emph{Exit Code} & \emph{Description}   \\ \hline \hline
0     & Normal exit of daemon                                  \\ \hline
99    & \Macro{DAEMON\_SHUTDOWN} evaluated to \Expr{True}      \\ \hline
\end{tabular}
\end{table}
\end{center}
