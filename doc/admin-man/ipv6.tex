%%%%%%%%%%%%%%%%%%%%%%%%%%%%%%%%%%%%%%%%%%%%%%%%%%%%%%%%%%%%%%%%%%%%%%%%%%%
\subsection{\label{sec:ipv6}Running HTCondor on an IPv6 Network Stack}
%%%%%%%%%%%%%%%%%%%%%%%%%%%%%%%%%%%%%%%%%%%%%%%%%%%%%%%%%%%%%%%%%%%%%%%%%%
\index{IPv6|(}

HTCondor supports using IPv4, IPv6, or both.  By default, HTCondor will
look at a machine's interfaces, and on that machine, enable each protocol
for which it finds at least one interface with an address of that procol.

To require IPv4, you may set \Macro{ENABLE\_IPV4} to true; if the
machine does not have an interface with an IPv4 address, HTCondor will
not start.  Likewise, to require IPv6, you may set \Macro{ENABLE\_IPV6}
to true.

If you set \Macro{ENABLE\_IPV4} to false, HTCondor will not use IPv4,
even if it is available; likewise for \Macro{ENABLE\_IPV6} and IPv6.

The default setting for \Macro{ENABLE\_IPV4} and \Macro{ENABLE\_IPV6} is
\Expr{auto}, which uses the corresponding protocol if and only HTCondor
finds an interface with an address of that protocol.

If both IPv4 and IPv6 networking are enabled, HTCondor runs in mixed mode.
In mixed mode, HTCondor daemons have at least one IPv4 address and at
least one IPv6 address.  Other daemons and the command-line tools choose
between these addresses based on which protocols are enabled for them; if
both are, they will prefer the first address listed by that daemon.

A daemon may be listening on one, some, or all of its machine's addresses.
(See \Macro{NETWORK\_INTERFACE}.)  Daemons may presently list at most two
addresses, one IPv6 and one IPv4.  Each address is the ``most public''
address of its protocol; by default, the IPv6 address is listed first.
HTCondor selects the ``most public'' address heuristically.

Nonetheless, there are two cases in which HTCondor may not use an IPv6
address when one is available:
\begin{itemize}
\item{When given a literal IP address, HTCondor will use that IP address.  }
\item{When looking up a host name using DNS, HTCondor will use the first
address whose protocol is enabled for the tool or daemon doing the look up.  }
\end{itemize}

You may force HTCondor to prefer IPv4 in all three of these situations by
setting the macro \Macro{PREFER\_IPV4} to true; this is the default.
With \Macro{PREFER\_IPV4} set, HTCondor daemons will list their
``most public'' IPv4 address first; prefer the IPv4 address when
choosing from another's daemon list; and prefer the IPv4 address when
looking up a host name in DNS.

In practice, both an HTCondor pool's central manager and any submit
machines within a mixed mode pool must have both IPv4 and IPv6 addresses
for both IPv4-only and IPv6-only \Condor{startd} daemons to function
properly.

\subsubsection{IPv6 and Host-Based Security}

You may freely intermix IPv6 and IPv4 address literals.  You may also specify
IPv6 netmasks as a legal IPv6 address followed by a slash followed by the
number of bits in the mask; or as the prefix of a legal IPv6 address followed
by two colons followed by an asterisk.  The latter is entirely equivalent to the
former, except that it only allows you to (implicitly) specify mask bits
in groups of sixteen.  For example, \texttt{fe8f:1234::/60} and
\texttt{fe8f:1234::*} specify the same network mask.

The HTCondor security subsystem resolves names in the ALLOW and DENY
lists and uses all of the resulting IP addresses.  Thus, to allow or deny
IPv6 addresses, the names must have IPv6 DNS entries (AAAA records), or
\MacroNI{NO\_DNS} must be enabled.

\subsubsection{IPv6 Address Literals}

When you specify an IPv6 address and a port number simultaneously, you
must separate the IPv6 address from the port number by placing square
brackets around the address.  For instance:

\begin{verbatim}
COLLECTOR_HOST = [2607:f388:1086:0:21e:68ff:fe0f:6462]:5332
\end{verbatim}

If you do not (or may not) specify a port, do not use the square brackets.
For instance:

\begin{verbatim}
NETWORK_INTERFACE = 1234:5678::90ab
\end{verbatim}

\subsubsection{IPv6 without DNS}

When using the configuration variable \Macro{NO\_DNS},
IPv6 addresses are turned into host names by taking the IPv6 address,
changing colons to dashes, and appending \MacroUNI{DEFAULT\_DOMAIN\_NAME}.
So,
\begin{verbatim}
2607:f388:1086:0:21b:24ff:fedf:b520
\end{verbatim}
becomes
\begin{verbatim}
2607-f388-1086-0-21b-24ff-fedf-b520.example.com
\end{verbatim}
assuming
\begin{verbatim}
DEFAULT_DOMAIN_NAME=example.com
\end{verbatim}

\index{IPv6|)}
