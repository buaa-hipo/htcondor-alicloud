%%%%%%%%%%%%%%%%%%%%%%%%%%%%%%%%%%%%%%%%%%%%%%%%%%%%%%%%%%%%%%%%%%%%%%
\subsection{\label{sec:CCB}HTCondor Connection Brokering (CCB)}
%%%%%%%%%%%%%%%%%%%%%%%%%%%%%%%%%%%%%%%%%%%%%%%%%%%%%%%%%%%%%%%%%%%%%%
\index{CCB (HTCondor Connection Brokering)}

HTCondor Connection Brokering, or CCB, is a way of allowing HTCondor
components to communicate with each other when one side is in a
private network or behind a firewall.  Specifically, CCB allows
communication across a private network boundary in the following
scenario: an HTCondor tool or daemon (process A) needs to connect to an
HTCondor daemon (process B), but the network does not allow a TCP
connection to be created from A to B; it only allows connections from
B to A.  In this case, B may be configured
to register itself with a CCB server that both A and B can connect to.
Then when A needs to connect to B, it can send a request to the CCB
server, which will instruct B to connect to A so that the two can
communicate.

As an example, consider an HTCondor execute node that is within
a private network. 
This execute node's \Condor{startd} is process B.
This execute node cannot normally run jobs submitted from a machine
that is outside of that private network, 
because bi-directional connectivity between the submit node and the
execute node is normally required.  
However, 
if both execute and submit machine can connect to the CCB server,
if both are authorized by the CCB server,
and if it is possible for the execute node within the private network
to connect to the submit node,
then it is possible for the submit node to run jobs on the
execute node.

To effect this CCB solution,
the execute node's \Condor{startd} within the private network
registers itself with the CCB
server by setting the configuration variable \Macro{CCB\_ADDRESS}.
The submit node's \Condor{schedd} communicates with the CCB server,
requesting that the execute node's \Condor{startd} open the TCP
connection.
The CCB server forwards this request to the execute node's \Condor{startd},
which opens the TCP connection.
Once the connection is open, bi-directional communication is enabled.

If the location of the execute and submit nodes is reversed 
with respect to the private network,
the same idea applies:
the submit node within the private network registers itself with a CCB server,
such that when a job is running and the execute node needs to connect back to
the submit node (for example, to transfer output files), 
the execute node can connect by going through CCB to request a connection.

If both A and B are in separate private networks, then CCB alone
cannot provide connectivity.  However, if an incoming port or port
range can be opened in one of the private networks, then the situation
becomes equivalent to one of the scenarios described above and CCB can
provide bi-directional communication given only one-directional
connectivity.  See section~\ref{sec:Port-Details} for information on
opening port ranges.  Also note that CCB works nicely with
\Condor{shared\_port}.

Unfortunately at this time, CCB does not support standard universe jobs.

Any \Condor{collector} may be used as a CCB server.  There is no
requirement that the \Condor{collector} acting as the CCB server
be the same \Condor{collector} that a daemon
advertises itself to (as with \MacroNI{COLLECTOR\_HOST}).
However, this is often a convenient choice.

\subsubsection{Example Configuration}

This example assumes that there is a pool of machines in a private
network that need to be made accessible from the outside,
and that the \Condor{collector} (and therefore CCB server)
used by these machines is accessible from the outside.
Accessibility might be achieved by
a special firewall rule for the \Condor{collector} port,
or by being on a dual-homed machine in both networks.

The configuration of variable \MacroNI{CCB\_ADDRESS} on
machines in the private network causes registration with
the CCB server as in the example:

\begin{verbatim}
  CCB_ADDRESS = $(COLLECTOR_HOST)
  PRIVATE_NETWORK_NAME = cs.wisc.edu
\end{verbatim}

The definition of \MacroNI{PRIVATE\_NETWORK\_NAME} ensures that all
communication between nodes within the private network continues to happen
as normal, and without going through the CCB server.
The name chosen for \MacroNI{PRIVATE\_NETWORK\_NAME} should be different
from the private network name chosen for any HTCondor installations that
will be communicating with this pool.

Under Unix, and with large HTCondor pools,
it is also necessary to give the \Condor{collector} acting as the CCB server
a large enough limit of file descriptors.
This may be accomplished with the configuration variable
\Macro{MAX\_FILE\_DESCRIPTORS} or an equivalent.
Each HTCondor process configured to use CCB with \MacroNI{CCB\_ADDRESS}
requires one persistent TCP connection to the CCB server.
A typical execute node
requires one connection for the \Condor{master},
one for the \Condor{startd},
and one for each running job, as represented by a \Condor{starter}.
A typical submit machine
requires one connection for the \Condor{master},
one for the \Condor{schedd},
and one for each running job, as represented by a \Condor{shadow}.
If there will be no administrative commands required
to be sent to the \Condor{master} from outside of
the private network, then CCB may be disabled in the \Condor{master}
by assigning \MacroNI{MASTER.CCB\_ADDRESS} to nothing:
\begin{verbatim}
  MASTER.CCB_ADDRESS =
\end{verbatim}

Completing the count of TCP connections in this example:
suppose the pool consists of 500 8-slot
execute nodes and CCB is not disabled in the configuration of the
\Condor{master} processes.
In this case, the count of needed file descriptors plus some extra
for other transient connections to the collector is
500*(1+1+8)=5000.
Be generous, and give it twice as many
descriptors as needed by CCB alone:

\begin{verbatim}
  COLLECTOR.MAX_FILE_DESCRIPTORS = 10000
\end{verbatim}

\subsubsection{Security and CCB}

The CCB server authorizes all daemons that register themselves with it
(using \Macro{CCB\_ADDRESS}) at the DAEMON authorization level (these
are playing the role of process A in the above description).  It
authorizes all connection requests (from process B) at the READ
authorization level.  As usual, whether process B authorizes process A
to do whatever it is trying to do is up to the security policy for
process B; from the HTCondor security model's point of view, it is as if
process A connected to process B, even though at the network layer,
the reverse is true.

\subsubsection{Troubleshooting CCB}

Errors registering with CCB or requesting connections via CCB are
logged at level \Dflag{ALWAYS} in the debugging log.
These errors may be identified by searching for "CCB" in the log message.
Command-line tools require the argument
\Opt{-debug} for this information to be visible.  To see details of
the CCB protocol add \Dflag{FULLDEBUG} to the debugging options for
the particular HTCondor subsystem of interest.
Or, add \Dflag{FULLDEBUG} to
\MacroNI{ALL\_DEBUG} to get extra debugging from all HTCondor
components.

A daemon that has successfully registered itself with CCB will
advertise this fact in its address in its ClassAd.  
The ClassAd attribute \Attr{MyAddress} will contain information
about its \AdStr{CCBID}.

\subsubsection{Scalability and CCB}

Any number of CCB servers may be used to serve a pool of HTCondor
daemons.  For example, half of the pool could use one CCB server and
half could use another.  Or for redundancy, all daemons could use both
CCB servers and then CCB connection requests will load-balance
across them.  Typically, the limit of how many daemons may be
registered with a single CCB server depends on the authentication
method used by the \Condor{collector} for DAEMON-level and READ-level access,
and on the amount of memory available to the CCB server.  We are not
able to provide specific recommendations at this time, 
but to give a very rough idea,
a server class machine should be able to handle CCB
service plus normal \Condor{collector} service for a pool containing
a few thousand slots without much trouble.

