%%%%%%%%%%%%%%%%%%%%%%%%%%%%%%%%%%%%%%%%%%%%%%%%%%%%%%%%%%%%%%%%%%%%%%%%%%%
\section{\label{sec:HTCondor-HDFS}Using HTCondor with the Hadoop File System}
%%%%%%%%%%%%%%%%%%%%%%%%%%%%%%%%%%%%%%%%%%%%%%%%%%%%%%%%%%%%%%%%%%%%%%%%%%%
\index{Hadoop Distributed File System (HDFS)!integrated with HTCondor}

The Hadoop project is an Apache project,
headquartered at \URL{http://hadoop.apache.org}, 
which implements an open-source, distributed file system across a large set
of machines.  
The file system proper is called the Hadoop File System, or HDFS,
and there are several Hadoop-provided tools which use the file system,
most notably databases and tools which use 
the map-reduce distributed programming style.  

Distributed with the HTCondor source code,
HTCondor provides a way to manage the daemons which implement an HDFS,
but no direct support for the high-level tools which run atop this file system.
There are two types of daemons, which together create an instance of 
a Hadoop File System.
The first is called the Name node, 
which is like the central manager for a Hadoop cluster.
There is only one active Name node per HDFS.
If the Name node is not running, no files can be accessed.
The HDFS does not support fail over of the Name node,
but it does support a hot-spare for the Name node,
called the Backup node.
HTCondor can configure one node to be running as a Backup node.
The second kind of daemon is the Data node,
and there is one Data node per machine in the distributed file system.
As these are both implemented in Java,
HTCondor cannot directly manage these daemons.
Rather, HTCondor provides a small DaemonCore daemon,
called \Condor{hdfs},
which reads the HTCondor configuration file, 
responds to HTCondor commands like \Condor{on} and \Condor{off},
and runs the Hadoop Java code.
It translates entries in the HTCondor configuration file 
to an XML format native to HDFS.
These configuration items are listed with the 
\Condor{hdfs} daemon in section~\ref{sec:HDFS-Config-File-Entries}. 
So, to configure HDFS in HTCondor,
the HTCondor configuration file should specify one machine in the
pool to be the HDFS Name node, and others to be the Data nodes.

Once an HDFS is deployed, 
HTCondor jobs can directly use it in a vanilla universe job,
by transferring input files directly from the HDFS by specifying 
a URL within the job's submit description file command
\SubmitCmd{transfer\_input\_files}. 
See section~\ref{sec:URL-transfer} for the administrative details
to set up transfers specified by a URL.
It requires that a plug-in is accessible and defined to handle
\Expr{hdfs} protocol transfers. 


%%%%%%%%%%%%%%%%%%%%%%%%%%%%%%%%%%%%%%%%%%%%%%%%%%%%%%%%%%%%%%%%%%%%%%%%%%%
\subsection{\label{sec:HDFS-Config-File-Entries}\condor{hdfs}
Configuration File Entries}
%%%%%%%%%%%%%%%%%%%%%%%%%%%%%%%%%%%%%%%%%%%%%%%%%%%%%%%%%%%%%%%%%%%%%%%%%%%$

\index{configuration!condor\_hdfs configuration variables}
These macros affect the \Condor{hdfs} daemon.
Many of these variables determine how the \Condor{hdfs} daemon sets
the HDFS XML configuration.

\begin{description}

\label{param:HDFSHome}
\item[\Macro{HDFS\_HOME}]
  The directory path for the Hadoop file system installation directory.
  Defaults to \File{\MacroUNI{RELEASE\_DIR}/libexec}.
  This directory is required to contain 

  \begin{itemize}
  \item{directory \File{lib}},
  containing all necessary jar files for the execution of a Name node
  and Data nodes.
  \item{directory \File{conf}},
  containing default Hadoop file system configuration files with names that
  conform to \Expr{*-site.xml}.
  \item{directory \File{webapps}},
  containing JavaServer pages (jsp) files for the Hadoop file 
  system's embedded server.
  \end{itemize}

\label{param:HDFSNamenode}
\item[\Macro{HDFS\_NAMENODE}]
  The host and port number for the HDFS Name node.
  There is no default value for this required variable.
  Defines the value of \Expr{fs.default.name} in the HDFS XML configuration.

\label{param:HDFSNamenodeWeb}
\item[\Macro{HDFS\_NAMENODE\_WEB}]
  The IP address and port number for the HDFS embedded web server within the
  Name node with the syntax of \Expr{a.b.c.d:portnumber}.
  There is no default value for this required variable.
  Defines the value of \Expr{dfs.http.address} in the HDFS XML configuration.

\label{param:HDFSDatanodeWeb}
\item[\Macro{HDFS\_DATANODE\_WEB}]
  The IP address and port number for the HDFS embedded web server within the
  Data node with the syntax of \Expr{a.b.c.d:portnumber}.
  The default value for this optional variable is 0.0.0.0:0, which means
  bind to the default interface on a dynamic port.
  Defines the value of \Expr{dfs.datanode.http.address} in 
  the HDFS XML configuration.

\label{param:HDFSNamenodeDir}
\item[\Macro{HDFS\_NAMENODE\_DIR}]
  The path to the directory on a local file system where the Name node will
  store its meta-data for file blocks.
  There is no default value for this variable; it is required to be defined
  for the Name node machine.
  Defines the value of \Expr{dfs.name.dir} in the HDFS XML configuration.

\label{param:HDFSDatanodeDir}
\item[\Macro{HDFS\_DATANODE\_DIR}]
  The path to the directory on a local file system where the Data node will
  store file blocks.
  There is no default value for this variable; it is required to be defined
  for a Data node machine.
  Defines the value of \Expr{dfs.data.dir} in the HDFS XML configuration.

\label{param:HDFSDatanodeAddress}
\item[\Macro{HDFS\_DATANODE\_ADDRESS}]
  The IP address and port number of this machine's Data node.
  There is no default value for this variable; it is required to be defined
  for a Data node machine, and may be given the value \Expr{0.0.0.0:0}
  as a Data node need not be running on a known port.
  Defines the value of \Expr{dfs.datanode.address} in the HDFS XML
  configuration.

\label{param:HDFS}
\item[\Macro{HDFS\_NODETYPE}] This parameter specifies the type of
  HDFS service provided by this machine.  Possible values are
  \Expr{HDFS\_NAMENODE} and \Expr{HDFS\_DATANODE}.  The default value
  is \Expr{HDFS\_DATANODE}.

\label{param:HDFSBackupnode}
\item[\Macro{HDFS\_BACKUPNODE}]
  The host address and port number for the HDFS Backup node.
  There is no default value.
  It defines the value of the HDFS dfs.namenode.backup.address
  field in the HDFS XML configuration file.

\label{param:HDFSBackupnodeWeb}
\item[\Macro{HDFS\_BACKUPNODE\_WEB}]
  The address and port number for the HDFS embedded web server 
  within the Backup node,
  with the syntax of hdfs://<host\_address>:<portnumber>.
  There is no default value for this required variable. 
  It defines the value of dfs.namenode.backup.http-address in the 
  HDFS XML configuration. 

\label{param:HDFSNamenodeRole}
\item[\Macro{HDFS\_NAMENODE\_ROLE}]
  If this machine is selected to be the Name node,
  then the role must be defined. 
  Possible values are \Expr{ACTIVE}, \Expr{BACKUP}, \Expr{CHECKPOINT}, 
  and \Expr{STANDBY}.
  The default value is \Expr{ACTIVE}.
  The \Expr{STANDBY} value exists for future expansion.
  If \MacroNI{HDFS\_NODETYPE} is selected to be Data node
  (\MacroNI{HDFS\_DATANODE}), then this variable is ignored.  

\label{param:HDFSLog4j}
\item[\Macro{HDFS\_LOG4J}]
  Used to set the configuration for the HDFS debugging level.
  Currently one of  \Expr{OFF}, \Expr{FATAL}, \Expr{ERROR}, \Expr{WARN},
  \Expr{INFODEBUG}, \Expr{ALL} or \Expr{INFO}.
  Debugging output is written to \File{\MacroUNI{LOG}/hdfs.log}.
  The default value is \Expr{INFO}.

\label{param:HDFSAllow}
\item[\Macro{HDFS\_ALLOW}]
  A comma separated list of hosts that are authorized with read and write
  access to the invoked HDFS.
  Note that this configuration variable name is likely to change to
  \MacroNI{HOSTALLOW\_HDFS}.

\label{param:HDFSDeny}
\item[\Macro{HDFS\_DENY}]
  A comma separated list of hosts that are denied access to the invoked HDFS.
  Note that this configuration variable name is likely to change to
  \MacroNI{HOSTDENY\_HDFS}.

\label{param:HDFSNamenodeClass}
\item[\Macro{HDFS\_NAMENODE\_CLASS}]
  An optional value that specifies the class to invoke.
  The default value is \verb@org.apache.hadoop.hdfs.server.namenode.NameNode@.

\label{param:HDFSDatanodeClass}
\item[\Macro{HDFS\_DATANODE\_CLASS}]
  An optional value that specifies the class to invoke.
  The default value is \verb@org.apache.hadoop.hdfs.server.datanode.DataNode@.

\label{param:HDFSSiteFile}
\item[\Macro{HDFS\_SITE\_FILE}]
  The optional value that specifies the HDFS XML configuration file to generate.
  The default value is \File{hdfs-site.xml}.

\label{param:HDFSReplication}
\item[\Macro{HDFS\_REPLICATION}]
  An integer value that facilitates setting the replication factor of an HDFS,
  defining the value of \Expr{dfs.replication} in the HDFS XML
  configuration.  This configuration variable is optional, as the HDFS has
  its own default value of 3 when not set through configuration.

\end{description}

